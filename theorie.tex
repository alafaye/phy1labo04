\subsection{Mouvement circulaire uniforme}
Ce type de mouvement est caractéristique des masses contraintes à faire une rotation le long d'un cercle en général de rayon constant.
La contrainte est modélisée par la force centripète qui pour un rayon fixe vaut:
\begin{equation}
    F=m\cdot \omega^2 \cdot r
\end{equation}
Ou $\omega$ représente la vitesse angulaire en $[rad/s]$, $m$ la masse en $[kg]$, et $r$ le rayon de rotation en $[m]$.
Comme $\omega=\frac{v}{r}$, on peut aussi noter:
\begin{equation}
    F=m\cdot\frac{v^2}{r}
\end{equation}
Sans cette force, la masse serait libérée de son mouvement et continuerait son parcours de manière rectiligne.
