\subsection{Mouvement circulaire uniforme}
Ce type de mouvement est caractéristique des masses contraintes à faire une rotation le long d'un cercle en général de rayon constant.
La contrainte est modélisée par la force centripète qui pour un rayon fixe vaut:
\begin{equation}
    F=m\cdot \omega^2 \cdot r
\end{equation}
Ou $\omega$ représente la vitesse angulaire en $[rad/s]$, $m$ la masse en $[kg]$, et $r$ le rayon de rotation en $[m]$.
Comme $\omega=\frac{v}{r}$, on peut aussi noter:
\begin{equation}
    F=m\cdot\frac{v^2}{r}
\end{equation}
Sans cette force, la masse serait libérée de son mouvement et continuerait son parcours de manière rectiligne.

\subsection{Incertitudes}

Une mesure expérimentale est toujours accompagnée de son $\textit{incertitude de mesure}$. On peut qualifier cette incertitude selon différentes caractéristiques:
\begin{itemize}
\item La résolution
\item La précision
\item Le reproductibilité
\end{itemize}

Cette incertitude a de multiples sources, humaines ou liées au matériel, qui la rendent inévitable mais pas pour autant non-quantifiable.
On la calcule en générale de deux manières, sous forme $\textit{d'incertitude absolue}$ ou $\textit{d'incertitude relative}$.

\paragraph{Notation}
Un résultat s'écrit donc sous la forme: \[a\pm\Delta a\] tel que $\Delta a$ soit l'incertitude absolue et $\frac{\Delta a}{a} \%$ soit l'incertitude relative.

\paragraph{Chiffres significatifs}
Pour noter correctement les résultats avec leur incertitude, il faut être attentif aux $\textit{chiffres significatifs}$, les mesures et résultats de calculs doivent être exprimés avec un ou x chiffres dont la valeur n'est pas certaine.\\
Par exemple, pour une mesure au gramme près $m = (2.3\pm0.1)kg$.

\paragraph{Propagation des erreurs}
Une fois l'incertitude estimée, il faut encore la propager aux tous les calculs qui suivront. Dans le cas d'une fonction à une seule variable, on peut utiliser sa différentielle:
\begin{equation}
    dg=|{\frac{df}{dx}}|dx
\end{equation}

Ce qui donne pratiquement:
\begin{equation}
    \Delta g=|\frac{df}{dx}|\Delta x
\end{equation}

Pour propager les erreurs durant les calculs, les formules suivantes sont utilisées:\\
Addition/Soustraction:
\begin{equation}
    \begin{split}
	g = a \pm b\\
	\Delta g = \Delta a + \Delta b
    \end{split}
\end{equation}
Multiplication/Division:
\begin{equation}
    \begin{split}
	g = k\cdot ab \;\;\; k=constante\\
	\frac{\Delta}{|g|} = \frac{\Delta}{|a|} + \frac{\Delta}{|b|}
    \end{split}
\end{equation}

\subsection{Régression linéaire}

Pour la regression linéaire, les formules suivantes peuvent être utilisées afin de calculer les incertitudes liées la courbe générée.
Dans ces équations, n représente le nombre de mesures et p la pente.

\begin{equation}
    \label{regressionr}
    r=\frac{\sum_{k=1}^{n}(x_k-\bar{x})\cdot(y_k-\bar{y})}{\sqrt{\sum_{k=1}^{n}(x_k-\bar{x})^2\cdot\sum_{k=1}^{n}(y_k-\bar{y})^2}}
\end{equation}
\\

\begin{equation}
    \label{deltam}
    \Delta p=p\sqrt{\frac{\frac{1}{r^2}-1}{n-2}}
\end{equation}
