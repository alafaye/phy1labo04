\paragraph{Buts}
Le but de cette expérience est de mesurer et d'analyser la force appliquée par une masse à un axe en rotation en faisant varier cette masse, la distance de cette masse au centre de rotation et la vitesse de rotation.

\paragraph{Matériel}
\begin{itemize}
    \item 3 masses de 50, 75 et 100 g.
    \item Une masse de contrepoids.
    \item Boitier d'acquisition Cassy®.
    \item Barrière infrarouge.
    \item Alimentation DC de laboratoire.
    \item Axe de rotation motorisé muni d'un capteur de force.
    \item Ordinateur
    \item Balance
\end{itemize}

\paragraph{Méthode}
Pour effectuer cette manipulation, le montage ci-dessous a été utilisé:
\begin{figure}[h]
    \caption[Photo du montage]{Photo du montage}
    \centering
    \includegraphics[height=21em]{photomanip.jpg}
\end{figure}
\\
Il est constitué d'une barre métallique attachée en son centre à un dynamomètre électronique captant la force dans l'axe de la barre sus-mentionée.
Cette barre est percée en plusieurs endroits afin d'y placer des masses cylindriques trouées en leur centre afin de les glisser le long de l'axe puis de les visser afin de les immobiliser.
Une masse de contre-poids est placée à l'extremité opposée afin de ne pas appliquer trop de force sur l'axe lorsque des masses trop imposantes sont placées en bout de barre.
Une barrière infrarouge est placé le long de la cage afin d'indiquer lorsque la barre en rotation coupe le faisceau dans le but de donner la vitesse angulaire $\omega$.

Pour effectuer les mesures, Cassy a été utilisé. C'est un logiciel accompagné d'un boîtier de mesure sur lequel le dynamomètre ainsi que la barrière infrarouge ont été branchés.
Ce boîtier est ensuite relié à un ordinateur sur lequel on a pu interagir avec les capteurs pour extraire leurs mesures

Pour chaque mesure le procédé suivant a été appliqué.
\begin{enumerate}
    \item À chaque changement de masse, vitesse angulaire ou rayon, le mobile a tourné à vide afin de remettre à zéro le capteur de force.
    \item Le masse a été solidement fixée à la distance désirée du centre.
    \item L'axe a été mis en rotation jusqu'à la vitesse angulaire souhaitée, avec quelques secondes d'attente pour être certain que le système se soit stabilisé.
    \item Prise de la mesure grâce à Cassy et ajout de la valeur dans le tableau.
\end{enumerate}

Dans l'objectif d'avoir des mesures les plus précises possibles les masses et leurs vis de fixation ont été précisément mesurées avec un balance ayant une précision au centigramme. Les valeurs suivantes ont été obtenues:
\begin{table}[ht]
    \caption[Mesure des masses]{Mesures des masses}
    \centering
    \begin{tabular}{|l|l|}
	\hline
	Masse théorique [kg] & Masse réelle [kg]\\
	\hline
	0.050 & $0.0517 \pm 0.01$\\
	0.075 & $0.0767 \pm 0.01$\\
	0.100 & $0.1017 \pm 0.01$\\
	\hline
    \end{tabular}
\end{table}

Les 60 mesures ont donc été efféctuées de cette manière, en variant la vitesse angulaire, ensuite la masse, et finalement le rayon. 

\paragraph{Resultats}
Les resultats bruts sont diponibles en annexe.

Pour l'analyse des données, la consignes de procéder en trois parties a été suivies.
\begin{itemize}
    \item Dans le premier cas, on observe la variation de la force uniquement en fonction du rayon
    \item Dans le second, on continue de regarder la force, mais en faisant varier la masse.
    \item Finalement on donne le rapport entre la force (toujours) et la vitesse angulaire au carré.
\end{itemize}

\newpage

%%%%%%%%%%%%%%%%%%%%%%%%%%%%%%%%%%%%%%%%%%%%%%%%%%%%%%%%%%%%%%%%%
\subsubsection{Force en fonction du rayon}

La vitesse angulaire constante choisie est de 25 rad/s car c’est là que les mesures sont les plus élevées.

\begin{table}[ht]
    \centering
    \caption[Tables mesures rayon force]{Tables des mesures choisies pour observer le rapport entre F et r}

    Masse m = 50g\\[1px]
    \begin{tabular}{|l|l|l|l|}
    \hline
	r [m]	&F [N]	&m [kg]	&$\omega$ [rad/s]\\
    \hline
	0,05	&1,52	&0,052	&25              \\
	0,1	&3,09	&0,052	&25              \\
	0,15	&4,62	&0,052	&25              \\
	0,2	&6,14	&0,052	&24,9            \\
	0,25	&7,68	&0,052	&24,9            \\
    \hline
    \end{tabular}\\[5px]

    Masse m = 75g\\[1px]
    \begin{tabular}{|l|l|l|l|}
    \hline
	r [m]	&F [N]	&m [kg]	&$\omega$ [rad/s]\\
    \hline
	0,05	&2,37	&0,077	&25,1            \\
	0,1	&4,65	&0,077	&24,9            \\
	0,15	&6,96	&0,077	&25,1            \\
	0,2	&9,75	&0,077	&25,1            \\
	0,25	&11,99	&0,077	&25              \\
    \hline
    \end{tabular}\\[5px]

    Masse m = 100g\\[1px]
    \begin{tabular}{|l|l|l|l|}
    \hline
	r [m]	&F [N]	&m [kg]	&$\omega$ [rad/s]\\
    \hline
	0,05	&3,11	&0,102	&25              \\
	0,1	&6,27	&0,102	&25              \\
	0,15	&9,44	&0,102	&25              \\
	0,2	&12,41	&0,102	&25,1            \\
	0,25	&15,23	&0,102	&25              \\
    \hline
    \end{tabular}
\end{table}

\begin{figure}[!h]
    \caption[Graphique rayon force]{Graphique de rapport entre le rayon et la force}
    \centering
    \includegraphics[height=20em]{graf1.png}
\end{figure}

Les courbes ont été forcées de passer par zéro afin de respecter la théorie et de faciliter les calculs d'incertitudes.

Pour chacune des courbes, les incertitudes ont éte calculées grâce à la formule~\eqref{deltam}, donnant ainsi les résultats suivants.

\begin{table}[ht]
    \centering
    \begin{tabular}{l l l l}
    m = 100g & p = 61,731 & $R^2$ = 0,9991 & $\Delta p$=1.069\\
    m = 75g  & p = 47,818 & $R^2$ = 0,9983 & $\Delta p$=1.139\\
    m = 50g  & p = 30,735 & $R^2$ = 0,9999 & $\Delta p$=0.177\\
    \end{tabular}
\end{table}

Pour la version théorique,l'équation~\eqref{force} a été utilisée, donc en utilisant dans ce cas:

\begin{equation}
    p' = \omega^2 \cdot m
\end{equation}

Comme les mesures de $\omega$ ne sont pas exactememt les mêmes dans chaque cas, on a d'abord calculé pour chacune de ces valeurs avant de faire la moyenne. Ce qui donne:

\begin{table}[ht]
    \centering
    \begin{tabular}{l l}
	m = 0.052kg & p' = 32.3962\\
	m = 0.077kg & p' = 48.2796\\
	m = 0.102kg & p' = 63.8522\\
    \end{tabular}
\end{table}

Pour calculer les incertitudes dans ce cas, la méthode des dérivées partielle a été sollicitée. Les valeurs ont été moyennées de manière analogue à celles ci-dessus.

\begin{equation}
    \Delta p' = |\frac{\delta p'}{\delta \omega^2}|\cdot \Delta \omega^2 + |\frac{\delta p'}{\delta m}|\cdot \Delta m
\end{equation}

Les valeurs théoriques avec incertitudes sont donc:

\begin{table}[ht]
    \centering
    \begin{tabular}{l l}
	m = 0.052kg & $p' \pm \Delta p'$ = 32.3962 $\pm$ 0.2658 \\
	m = 0.077kg & $p' \pm \Delta p'$ = 48.2796 $\pm$ 0.3919 \\
	m = 0.102kg & $p' \pm \Delta p'$ = 63.8522 $\pm$ 0.5167 \\
    \end{tabular}
\end{table}

\paragraph{Conclusion}

\newpage
%%%%%%%%%%%%%%%%%%%%%%%%%%%%%%%%%%%%%%%%%%%%%%%%%%%%%%%%%%%%%%%%%
\subsubsection{Force en fonction de la masse}

\begin{table}[ht]
    \centering
    \caption[Tables mesures masse force]{Tables des mesures choisies pour observer le rapport entre F et m}

    Rayon r = 0.05m\\[1px]
    \begin{tabular}{|l|l|l|}
    \hline
	$\omega$ [rad/s] &F [N]	&m [kg]	\\
    \hline
	25	&1,52	&0,052 \\
	25,1	&2,37	&0,077 \\
	25	&3,11	&0,102 \\
    \hline
    \end{tabular}\\[5px]

    Rayon r = 0.15m\\[1px]
    \begin{tabular}{|l|l|l|}
    \hline
	$\omega$ [rad/s] &F [N]	&m [kg]	\\
    \hline
	25	&4,62	&0,052 \\
	25,1	&6,96	&0,077 \\
	25	&9,44	&0,102 \\
    \hline
    \end{tabular}\\[5px]

    Rayon r = 0.25m\\[1px]
    \begin{tabular}{|l|l|l|}
    \hline
	$\omega$ [rad/s] &F [N]	&m [kg]	\\
    \hline
	24,9	&7,68	&0,052 \\
	25	&11,99	&0,077 \\
	25	&15,23	&0,102 \\
    \hline
    \end{tabular}
\end{table}

\begin{figure}[!h]
    \caption[Graphique masse force]{Graphique de rapport entre la masse et la force}
    \centering
    \includegraphics[height=22em]{graf2.png}
\end{figure}

\begin{table}[ht]
    \centering
    \begin{tabular}{l l l l}
    r = 0.25m & p = 151,08& $R^2$ = 0,9933 & $\Delta p$=12.41\\
    r = 0.15m & p = 91,351& $R^2$ = 0,9968 & $\Delta p$=5.176\\
    r = 0.05m & p = 30,401& $R^2$ = 0,9963 & $\Delta p$=1.853\\
    \end{tabular}
\end{table}

Dans le but de calculer les valeurs théoriques, l'équation~\eqref{force} a encore été utilisée:

\begin{equation}
    p' = \omega^2 \cdot r
\end{equation}

Comme dans le cas précédent, il y a de legéres différences entre les valeurs de $\omega^2$, une moyenne a de nouveau été faite aussi bien pour les valeurs de $p'$ que pour celles de $\Delta p'$.

\begin{table}[ht]
    \centering
    \begin{tabular}{l l}
	r = 0.25m & p' = 31.3335 \\
	r = 0.15m & p' = 94.0005 \\
	r = 0.05m & p' = 155.834 \\
    \end{tabular}
\end{table}

Le procédé pour le calcul des incertitudes est exactement le même que pour le cas où le rayon variait.

\begin{equation}
    \Delta p' = |\frac{\delta p'}{\delta \omega^2}|\cdot \Delta \omega^2 + |\frac{\delta p'}{\delta r}|\cdot \Delta r
\end{equation}

Finalement:

\begin{table}[ht]
    \centering
    \begin{tabular}{l l}
	r = 0.25m & $p' \pm \Delta p'$ = 31.3335 $\pm$ 0.2503 \\
	r = 0.15m & $p' \pm \Delta p'$ = 94.0005 $\pm$ 0.7510 \\
	r = 0.05m & $p' \pm \Delta p'$ = 155.834 $\pm$ 1.2483 \\
    \end{tabular}
\end{table}

\paragraph{Conclusion}

\newpage
\clearpage
%%%%%%%%%%%%%%%%%%%%%%%%%%%%%%%%%%%%%%%%%%%%%%%%%%%%%%%%%%%%%%%%%
\subsubsection{Force en fonction du carré de la vitesse angulaire}

Pour cette partie, les rayons ont été choisis les plus différents possibles afin d'augmenter la dispersion des résultats  que voici.

\begin{table}[ht]
    \caption[Tables mesures vitesse angulaire force]{Tables des mesures choisies pour observer le rapport entre F et $\omega^2$}
    \centering

    Rayon r=0.05m\\[1px]
    \begin{tabular}{|l|r|l|l|}
	\hline
	$\omega^2$ [rad/s] & Force [N] & Masse [kg] & Rayon [m]\\
	\hline
	098.01	&0.51	&0.102	&0.05\\
	219.04	&1.16	&0.102	&0.05\\
	400	&2.09	&0.102	&0.05\\
	625	&3.11	&0.102	&0.05\\
	\hline
    \end{tabular}\\[5px]

    Rayon r=0.15m\\[1px]
    \begin{tabular}{|l|r|l|l|}
	\hline
	$\omega^2$ [rad/s] & Force [N] & Masse [kg] & Rayon [m]\\
	\hline
	100	&1.41	&0.102	&0.15\\
	228.01	&3.14	&0.102	&0.15\\
	400	&6.24	&0.102	&0.15\\
	625	&9.44	&0.102	&0.15\\
	\hline
    \end{tabular}\\[5px]

    Rayon r=0.25m\\[1px]
    \begin{tabular}{|l|r|l|l|}
	\hline
	$\omega^2$ [rad/s] & Force [N] & Masse [kg] & Rayon [m]\\
	\hline
	100	&2.51	&0.102	&0.25\\
	225	&5.24	&0.102	&0.25\\
	400	&9.48	&0.102	&0.25\\
	625	&15.23	&0.102	&0.25\\
	\hline
    \end{tabular}
\end{table}

\begin{figure}[!h]
    \caption[Graphique $\omega^2$ force]{Graphique de rapport entre la vitesse angulaire au carré et la force}
    \centering
    \includegraphics[height=22em]{graf3.png}
\end{figure}

Pour les incertitudes, l'équation~\eqref{deltam} a été utilisée.\\
A.N.:
\begin{equation}
    0.0241\sqrt{\frac{\frac{1}{0.9989^2}-1}{n-2}}=0.0006
\end{equation}

On procède de la même manière pour les autres valeurs et l'on obtient:

\begin{table}[ht]
    \centering
    \begin{tabular}{l l l l}
    r = 0.25m & p = 0,0241 & $R^2$ = 0,9989& $\Delta p$=0.0006\\
    r = 0.15m & p = 0,0151 & $R^2$ = 0,9962& $\Delta p$=0.0007\\
    r = 0.05m & p = 0,0051 & $R^2$ = 0,9974& $\Delta p$=0.0002\\
    \end{tabular}
\end{table}

Pour le calcul avec la formule théorique~\eqref{force} on obtient:

\begin{equation}
    p' = m \cdot r
\end{equation}

A.N.:

\begin{table}[ht]
    \centering
    \begin{tabular}{l l}
	r = 0.05m & p' = 0.0051\\
	r = 0.15m & p' = 0.0153\\
	r = 0.25m & p' = 0.0255\\
    \end{tabular}
\end{table}

Pour les incertitudes il faut utiliser~\eqref{multdiv}, ce qui nous permet d'obtenir:

\begin{equation}
    \Delta p' = p' \cdot (\frac{\Delta m}{m} + \frac{\Delta r}{r})
\end{equation}

La barre de métal étant finement ouvragée, on suppose $\Delta r$ négligeable, donc:

\begin{equation}
    \Delta p' = p' \cdot \frac{\Delta m}{m}
\end{equation}

Les valeurs finales sont donc:

\begin{table}[ht]
    \centering
    \begin{tabular}{l l}
	r = 0.05m & $p'\pm \Delta p'$ = 0.0051 $\pm$ 0.0001\\
	r = 0.15m & $p'\pm \Delta p'$ = 0.0153 $\pm$ 0.0002\\
	r = 0.25m & $p'\pm \Delta p'$ = 0.0255 $\pm$ 0.0003\\
    \end{tabular}
\end{table}

\paragraph{Conclusion}

On peut donc raisonnablement affirmer que la force est linéairement proportionelle au carré de la vitesse angulaire.
Les mesures effectuées correspondent très bien à la théorie et les incertitudes sont donc très réduites.
